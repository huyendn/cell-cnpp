\documentclass[10pt]{article}
\usepackage[hmargin={0.6in, 0.6in}, vmargin={0.9in, 0.9in}]{geometry}
\usepackage{amsmath, amsthm, booktabs}
\usepackage{amssymb}
\usepackage{pdfpages}
\usepackage{setspace}
\usepackage{booktabs}
\usepackage{graphicx}

%\usepackage{fancyheadings}
%\usepackage{hyperref}

\renewcommand{\baselinestretch}{1}

\newtheorem{proposition}{Proposition}

\newcommand{\dif}{\mathrm{d}}
\newcommand{\Var}{\mathbb{V}}
\newcommand{\PP}{\mathbb{P}}
\newcommand{\EE}{\mathbb{E}}

\title{Modeling the Cell Proliferation using Compound Non-homogeneous Poisson Distribution}
\author{The Author}
\date{} % Activate to display a given date or no date (if empty),
         % otherwise the current date is printed
\begin{document}
\maketitle
\section{Model}
The evolution of the average cell counts of the active stem cell, $S(t)$, and differentiated cells, $F(t)$, is
described by the following differential equations
\begin{equation}\label{count*ODE}
\begin{split}
\frac{dS(t)}{dt} &= r[p_1(t) - p_3(t)] S(t), \quad S(0)=S_0, \\
\frac{dF(t)}{dt} &= r[p_2(t) + 2p_3(t)] S(t), \quad F(0)=0,
\end{split}
\end{equation}
where $r>0$ and  $p_1(t), p_2(t), p_3(t)>0$ and  $0<p_1(t)+p_2(t)+p_3(t)\leq 1$ for any $t>0$. The solutions
of the differential equations are given by
\begin{equation*}
\begin{split}
S(t) & = S_0 \exp \left\{ r P(t) \right\}, \\
F(t) & = \int_0^t r  [p_2(v) + 2p_3(v)] S(v) dv\\
&= S_0  r   \int_0^t [p_2(v) + 2p_3(v)] \exp \left\{r P(v) \right\} dv,
\end{split}
\end{equation*}
where $P(t) = \int_0^t [p_1(v) - p_3(v)] dv$.

We model the stem cell and differentiated cell counts using the compound nonhomogeneous Poisson process as follows.
Let $N(t)$ be a nonhomogeneous Poisson process with rate $\lambda(t) = r S(t)$. Given a sequence of arrival times $S_1, S_2, S_3,\dots$, the pairs of random variables
$(X_1,Y_1), (X_2,Y_2), (X_3,Y_3), \dots$ are independent with the following marginal distributions
\begin{equation*}
(X_k, Y_k) = \begin{cases}
(+1,0),  & p_1(S_k) \\
(0,+1),  & p_2(S_k) \\
(-1,+2), & p_3(S_k) \\
(0,0),   & p_4(S_k)\\
\end{cases},
\end{equation*}
where $p_4(t)=1-p_1(t)-p_2(t)-p_3(t).$
First, we define two compound nonhomogeneous Poisson processes, $X(t)$ and $Y(t)$, as follows
\begin{equation*}
X(t) = \sum_{k=1}^{N(t)} X_{k}, \quad \text{and} \quad
Y(t) = \sum_{k=1}^{N(t)} Y_{k}.
\end{equation*}
Next, we define the stopping time $\tau$ (when the stem cell count reaches 0) by
\begin{equation*}
\tau=\inf\{t>0:X(t)=-S_0\}.
\end{equation*}
Under assumption $\PP(\tau<\infty)$ the counts of stem cells and differentiated cells are modelled by processes $S_0+X(t\wedge\tau)$ and $Y(t\wedge\tau)$, respectively.

\section{Some Observations}

From Theorem 2.1 and Corollary to Proposition 2.2 of Chen and Savits (1993) (cite), the characteristic functions, expected values, and variances of $X(t)$ and $Y(t)$ are
given by
\begin{equation}\label{Chen*Savits*formulas}
\begin{split}
\EE[e^{iuX(t)}] &= \exp \left\{ \int_0^t \Big[ \left(e^{iu}-1\right)p_2(v) + \left(e^{-iu} - 1\right)p_3(v) \Big] r S(v) dv\right\},\\
\EE[X(t)] &= \int_0^t [p_1(v) - p_3(v)] r S(v) dv,\\
\Var(X(t)) &= \int_0^t [p_1(v) + p_3(v)] r S(v) dv, \\
\EE[e^{iuY(t)}] &= \exp \left\{ \int_0^t \Big[ \left(e^{iu}-1\right)p_2(v) + \left(e^{2iu} - 1\right)p_3(v) \Big] r S(v) dv\right\},\\
\EE[Y(t)] &= \int_0^t [p_2(v) + 2p_3(v)] r S(v) dv,\\
\Var(Y(t)) &= \int_0^t [p_2(v) + 4p_3(v)] r S(v) dv.
\end{split}
\end{equation}
Differential equations~(\ref{count*ODE}) immediately tell us that 
\begin{equation*}
\EE[X(t)] = \int_0^t [p_1(v) - p_3(v)]   r  S(v) dv=\int_0^t dS(v)=S(t)-S_0,
\end{equation*}
or $\EE[S_0+X(t)] = S(t)$. 
Similarly,
\begin{equation*}
\EE[Y(t)] = \int_0^t [p_2(v) + 2p_3(v)] r S(v) dv=\int_0^t dF(v)=F(t).
\end{equation*}

Moreover, when the initial number of stem cells is large, for any fixed time point $t$ the both stochastic processes are close
to their expected values. More specifically, we have the following two limit results.
\begin{proposition}
For a fixed $t$, $[S_0+X(t)]/S(t)  \overset{P}{\longrightarrow} 1$ and $Y(t)/F(t) \overset{P}{\longrightarrow} 1$ as $S_0 \rightarrow \infty$.
\end{proposition}
\begin{proof}
Using  formulas~(\ref{Chen*Savits*formulas}) and the first-order Taylor's expansion for the exponential function $e^{iu/S(t)}=e^{iu/(S_0 \exp \left\{ r P(t) \right\})}$ we get
\begin{equation*}
\begin{split}
\EE\Big[e^{iu \frac{S_0+X(t)}{S(t)}}\Big] & =e^{iu S_0/S(t)} \exp \left\{  \int_0^t \Big[ \left(e^{i u/S(t)} - 1\right) p_1(v) + \left(e^{-i u/S(t)} - 1\right)p_3(v)\Big] r S(v) dv\right\} \\
& = e^{iu S_0/S(t)}\exp \left\{ \int_0^t \Big[ \Big(\frac{iu}{S(t)}  + o\left(S_0^{-1}\right) \Big) p_1(v) + \Big( - \frac{iu}{S(t)}  + o\left(S_0^{-1}\right) \Big) p_3(v)  \Big] r S_0\exp\{r P(v)\} dv\right\} \\
& = e^{iu S_0/S(t)}\exp \left\{ \frac{iu}{S(t)} \int_0^t  [p_1(v) - p_3(v)] r S(v) dv  + o(1)\right\}\\       % o (S_0^{-1})\int_0^t [p_1(v) + p_3(v)] r S(v) dv \right\} \\
& = e^{iu S_0/S(t)}\exp \left\{ \frac{iu}{S(t)} \EE(X(t))  +  o (1)\right\}\\                                  % o(1)\int_0^t [p_1(v) + p_3(v)] r \exp\{r P(v)\} dv \right\} \\
& = \exp \left\{ \frac{iu}{S(t)} [S_0+\EE(X(t))]  +  o (1)\right\} \\                                                  %0(1) \int_0^t [p_1(v) + p_3(v)] r \exp\{r P(v)\} dv \right\} \\
& \rightarrow \exp \left\{iu\right\},
\end{split}
\end{equation*}
as $S_0\to\infty.$
Thus, $[S_0+X(t)]/S(t)  \overset{D}{\longrightarrow} 1$, and, therefore, $[S_0+X(t)]/S(t)  \overset{P}{\longrightarrow} 1$. Similarly, we get the convergence for $Y(t)$.
\end{proof}
In the same fashion a stronger, Central Limit Theorem–type results can also be obtained.
\begin{proposition}
For a fixed $t$, as $S_0 \rightarrow \infty$,
\begin{equation*}
\frac{S_0+X(t) - S(t)}{\sqrt{\text{Var}(X(t))}} \overset{D}{\longrightarrow} N(0, 1),
\end{equation*}
and
\begin{equation*}
\frac{Y(t) - F(t)}{\sqrt{\Var(Y(t))}} \overset{D}{\longrightarrow} N(0, 1).
\end{equation*}
\end{proposition}
\begin{proof}
Let $\sigma_{x_t} = \sqrt{\Var(X(t))}$. Using the second-order Taylor's expansion we have
\begin{equation*}
\begin{split}
\EE \Big[e^{iu X(t)/\sigma_{x_t}} \Big] &= \exp \left\{  \int_0^t \Big[ \left(e^{i u/\sigma_{x_t}} - 1\right)p_1(v) + \left(e^{-i u/\sigma_{x_t}} - 1\right)p_3(v)\Big] r S(v) dv\right\} \\
& = \exp \left\{ \int_0^t \Big[ \Big( \frac{iu}{\sigma_{x_t}} + \frac{i^2 u^2}{2 \sigma^2_{x_t}} + o\left(\sigma^{-2}_{x_t}\right) \Big) p_1(v) +  \Big(- \frac{iu}{\sigma_{x_t}} + \frac{i^2 u^2}{2 \sigma^2_{x_t}} + o\left(\sigma^{-2}_{x_t}\right) \Big) p_3(v)  \Big] r S(v) dv\right\} \\
& = \exp \left\{ \int_0^t \Big[ \Big( \frac{iu}{\sigma_{x_t}} + \frac{i^2 u^2}{2 \sigma^2_{x_t}} + o\left(S_0^{-1}\right) \Big) p_1(v) +  \Big(- \frac{iu}{\sigma_{x_t}} + \frac{i^2 u^2}{2 \sigma^2_{x_t}} + o\left(S_0^{-1}\right) \Big) p_3(v)  \Big] r S_0\exp\{r P(v)\} dv\right\} \\
& = \exp \left\{ \frac{iu}{\sigma_{x_t}} \int_0^t  [p_1(v) - p_3(v)] r S(v) dv + \frac{i^2u^2}{2\sigma^2_{x_t}}\int_0^t  [p_1(v) + p_3(v)] r S(v) dv  +  o (1) \right\} \\
& = \exp \left\{ \frac{iu}{\sigma_{x_t}} [S(t)-S_0] + \frac{i^2u^2}{2\sigma^2_{x_t}}\Var(X(t)) +  o (1)\right\}. \\
\end{split}
\end{equation*}
Therefore, as $S_0\to \infty$ we have
$$
\EE\Big[ e^{iu \frac{S_0+X(t)-S(t)} {\sigma_{x_t}}} \Big]= \exp \left\{ -\frac{u^2}{2} +  o (1)\right\} \rightarrow e^{- \frac{u^2}{2 }}.
$$
Again, the convergence for $Y(t)$ can be shown in a similar way.
\end{proof}

\newpage

\section{Covariance of $X(t)$ and $Y(t)$ when $p_4(t) = 0$}

When $p_4(t) = 0 \forall t$, i.e all the stem cells can undergo further division and there is no inactive stem cell, the number of divisions equal to the sum of stem cells and differentiated cells $N(t) = X(t) + Y(t)$. For each division occurrence, the sum of stem cells and ependymal cells increases by 1 (Table \ref{tab:cellchange}).
\begin{table}[!h]
\begin{center}
\begin{tabular}{ |c|c|c|c| }
 \hline
  & $p_1(t)$ & $p_2(t)$ & $p_3(t)$ \\
\hline
 $X(t)$ & $+1$ & $+0$ & $-1$ \\
 $Y(t)$ & $+0$ & $+1$ & $+2$ \\
 \hline
$X(t) + Y(t)$ & $+1$ & $+1$& $+1$\\
\hline
\end{tabular}
\end{center}
\caption{Change of total cell counts with each division occurrence when $p_4(t) = 0$.}
\label{tab:cellchange}
\end{table}
Then, $E[(N(t))^2] = E[(X(t))^2] + E[X(t)Y(t)] + E[(Y(t))^2]$. Since $N(t)$ is an nonhomogeneous Poisson process with rate $\lambda(t) = r S(t)$, $N(t)$ is a Poisson random variable with mean and variance
$$E[N(t)]=Var(N(t))=\int_0^t r S(v)dv.$$
We obtain the covariance of $X(t)$ and $Y(t)$ when $p_4(t) = 0$

\textcolor{red}{Correction}
\begin{equation}
\begin{split}
Cov(X(t), Y(t)) & =  E[X(t)Y(t)] -E[X(t)] E[Y(t)]  \\
& = E[(N(t))^2] - E[(X(t))^2] - E[(Y(t))^2] - E[X(t)] E[Y(t)]   \\
& = \Big[(E[N(t)])^2 + Var(N(t))  - (E[X(t)])^2 - Var(X(t))  - (E[Y(t)])^2 - Var(Y(t))\Big]\textcolor{red}{/2} \\
& - E[X(t)] E[Y(t)] \\
& \textcolor{red}{=\Big[(E[N(t)])^2 + Var(N(t))   - Var(X(t))   - Var(Y(t))- (E[X(t)])^2- (E[Y(t)])^2- 2E[X(t)] E[Y(t)]\Big]/2}\\
& \textcolor{red}{\text{(rearranging terms)}} \\
& \textcolor{red}{=\Big[(E[N(t)])^2 + Var(N(t))   - Var(X(t))   - Var(Y(t))- (E[N(t)])^2\Big]/2}\\
& \textcolor{red}{=\Big[\int_0^t r S(v) dv - \int_0^t  [p_1(v) +p_3(v)] r S(v) dv - \int_0^t  p_2(v) + 4p_3(v)] r S(v) dv\Big]/2}  \\
& \textcolor{red}{=- \int_0^t 2p_3(v) r S(v) dv.}
\end{split}
\end{equation}

\textbf{Limit of the $X(t)$ and $Y(t)$}

Show the variance of $X(t)$ and of $Y(t)$ converge as $t$ is large.
$$Var(X(t)) = \int_0^t [p_1(v)+p_3(v)] r S(v) dv,$$
$$Var(Y(t)) = \int_0^t [p_2(v) + 4p_3(v)] r S(v) dv.$$

Since $p_1(v) + p_3(v) \leq 1$ for any $v \in (0, \infty)$, then $[p_1(v)+p_3(v)] r S(v) \leq rS(v)$. Then using the comparison test if $\int_0^\infty r S(v) dv$ converges then $\int_0^\infty [p_1(v)+p_3(v)] r S(v) dv$ converges

(Note that in the compound nonhomogeneous Poisson process, $E[N(t)] = Var(N(t)) = \int_0^t r S(v) dv$).
\begin{equation}
\begin{split}
\int_0^\infty r S(v) dv & = r \int_0^\infty S_0 \exp \left\{ r \int_0^v [p_1(u) - p_3(u)] du \right\} dv
\end{split}
\end{equation}
Assume that $p_1(u) - p_3(u)$ is a continuous function on $(0, \infty)$ and there exists $u_0 \in (0, \infty)$ such that $\forall u>u_0, p_1(u) - p_3(u) \leq -q < 0$,
\begin{equation}
\begin{split}
\int_0^v [p_1(u) - p_3(u)] du  &= \int_0^{u_0}  [p_1(u) - p_3(u)] du  + \int_{u_0}^{v}  [p_1(u) - p_3(u)] du  \\
& \leq  \int_0^{u_0}  1 du + \int_{u_0}^{v} (-q) du  \quad \text{since $p_1(u)-p_3(u) \leq 1 \forall u$} \\
& = u_0 - rq(v-u_0).
\end{split}
\end{equation}
Then,
\begin{equation}
\begin{split}
\int_0^\infty r S(v) dv & = r \int_0^\infty S_0 \exp \left\{ r \int_0^v [p_1(u) - p_3(u)] du \right\} dv \\
& \leq r \int_0^\infty S_0 \exp \{r ( u_0 - rq(v-u_0))\} dv \\
& = r \int_0^\infty S_0 \exp\{u_0 (1+rq)\} \exp\{-rqv\}dv \\
& = S_0 \cdot r \cdot \exp\{u_0 (1+rq)\} \int_0^\infty\exp\{-rqv\}dv \\
& = \frac{S_0 \exp\{u_0 (1+rq)\}}{q} < \infty.
\end{split}
\end{equation}
Similarly for $Var(Y(t))$, since $[p_2(v) + 4p_3(v)] \leq 4$ for any $v \in (0, \infty)$,
$$ \int_0^\infty [p_2(v) + 4p_3(v)] r S(v) dv \leq \int_0^\infty 4 r S(v) dv  \leq \frac{4 S_0 \exp\{u_0 (1+rq)\}}{q} < \infty.$$


\section*{Results from the Multivariate Characteristic Function}
We model the stem cell (both viable and non-viable) and differentiated cell counts using the compound nonhomogeneous Poisson process as follows. Let $N(t)$ be a nonhomogeneous Poisson process with rate $\lambda(t) = r S(t)$. Given a sequence of arrival times $S_1, S_2, S_3, \cdots$, the 3-tuple of random variable $(X_1, Y_1, Z_1), (X_2, Y_2, Z_2), (X_3, Y_3, Z_3), \cdots$ are independent with the following marginal distributions
\begin{equation*}
(X_k, Y_k, Z_k) = \begin{cases}
(+1,0,0), & p_1(S_k) \\
(0,+1,0), & p_2(S_k) \\
(-1,+2,0), & p_3(S_k) \\
(0,0,+1), & p_4(S_k)
\end{cases}
\end{equation*}
where $p_4(t) = 1-p_1(t)-p_2(t)-p_3(t)$. We define the three compound nonhomogeneous Poisson processes, $X(t), Y(t),$ and $Z(t)$, as follows
$$X(t) = \sum_{k=1}^{N(t)} X_k, \quad Y(t) = \sum_{k=1}^{N(t)} Y_k, \quad \text{and} \quad Z(t) = \sum_{k=1}^{N(t)} Z_k.$$
From Theorem 2.1 of Chen and Savits (1993) (cite), the characteristic functions of $\mathbf{C}(t) = (X(t), Y(t), Z(t))$ is
\begin{equation}
\begin{split}
\psi_{\mathbf{C}(t)}(\mathbf{u}) & =  \exp \Big\{ \int_0^t [(e^{iu_1}-1)p_1(v) + (e^{iu_2}-1)p_2(v) + (e^{-iu_1+2iu_2}-1)p_3(v) + (e^{iu_3}-1)p_4(v)]  rS(v)dv \Big\}
\end{split}
\end{equation}
We use Leibniz integral rule to take the first and second derivative (using the product rule) of $\psi_{\mathbf{C}(t)}(\mathbf{u})$ with respect to $u_1, u_2,$ and $u_3$, we obtain the expected values and variances of $X(t), Y(t),$ and $Z(t)$ 
\begin{equation}
\begin{split}
\frac{\partial \psi_{\mathbf{C}(t)}(\mathbf{u})}{\partial u_1} &= \exp \Big\{ \int_0^t [(e^{iu_1}-1)p_1(v) + (e^{iu_2}-1)p_2(v) + (e^{-iu_1+2iu_2}-1)p_3(v) + (e^{iu_3}-1)p_4(v)]  rS(v)dv \Big\} \\
&\int_0^t [ie^{iu_1}p_1(v) - ie^{-iu_1 + 2iu_2}p_3(v)]  rS(v)dv \\
\frac{\partial^2 \psi_{\mathbf{C}(t)}(\mathbf{u})}{\partial^2 u_1} &= \exp \Big\{ \int_0^t [(e^{iu_1}-1)p_1(v) + (e^{iu_2}-1)p_2(v) + (e^{-iu_1+2iu_2}-1)p_3(v) + (e^{iu_3}-1)p_4(v)]  rS(v)dv \Big\} \\
& \int_0^t [i^2 e^{iu_1}p_1(v) + i^2 e^{-iu_1 + 2iu_2} p_3(v)]  rS(v)dv +\\
& \exp \Big\{ \int_0^t [(e^{iu_1}-1)p_1(v) + (e^{iu_2}-1)p_2(v) + (e^{-iu_1+2iu_2}-1)p_3(v) + (e^{iu_3}-1)p_4(v)]  rS(v)dv \Big\} \\
& \Big[\int_0^t [ie^{iu_1}p_1(v) - ie^{-iu_1 + 2iu_2}p_3(v)] rS(v)dv \Big]^2
\end{split}
\end{equation}

\begin{equation}
\begin{split}
\EE[X(t)] & = i^{-1} \frac{\partial \psi_{\mathbf{C}(t)}(\mathbf{u})}{\partial u_1} \Big \vert_{\mathbf{u} = \mathbf{0}} = \int_0^t [p_1(v)-p_3(v)] r S(v)dv \\
\EE[X(t)^2] & = i^{-2} \frac{\partial^2 \psi_{\mathbf{C}(t)}(\mathbf{u})}{\partial^2 u_1} \Big \vert_{\mathbf{u}= \mathbf{0}} \\
& = \int_0^t [p_1(v)+p_3(v)] r S(v)dv + \Big[\int_0^t [p_1(v)-p_3(v)] r S(v)dv\Big]^2 \\
\Var[X(t)] & = \EE[X(t)^2] - (\EE[X(t)])^2 = \int_0^t [p_1(v)+p_3(v)] r S(v)dv
\end{split}
\end{equation}

\begin{equation}
\begin{split}
\frac{\partial \psi_{\mathbf{C}(t)}(\mathbf{u})}{\partial u_2} &= \exp \Big\{ \int_0^t [(e^{iu_1}-1)p_1(v) + (e^{iu_2}-1)p_2(v) + (e^{-iu_1+2iu_2}-1)p_3(v) + (e^{iu_3}-1)p_4(v)]  rS(v)dv \Big\} \\
&\int_0^t [ie^{iu_2}p_2(v) + 2ie^{-iu_1 + 2iu_2}p_3(v)]  rS(v)dv \\
\frac{\partial^2 \psi_{\mathbf{C}(t)}(\mathbf{u})}{\partial^2 u_2} &= \exp \Big\{ \int_0^t [(e^{iu_1}-1)p_1(v) + (e^{iu_2}-1)p_2(v) + (e^{-iu_1+2iu_2}-1)p_3(v) + (e^{iu_3}-1)p_4(v)]  rS(v)dv \Big\} \\
& \int_0^t [i^2 e^{iu_2}p_2(v) +4i^2 e^{-iu_1 + 2iu_2} p_3(v)] rS(v)dv +\\
& \exp \Big\{ \int_0^t [(e^{iu_1}-1)p_1(v) + (e^{iu_2}-1)p_2(v) + (e^{-iu_1+2iu_2}-1)p_3(v) + (e^{iu_3}-1)p_4(v)]  rS(v)dv \Big\} \\
& \Big[\int_0^t [ie^{iu_2}p_2(v) + 2ie^{-iu_1 + 2iu_2}p_3(v)]  rS(v)dv \Big]^2
\end{split}
\end{equation}

\begin{equation}
\begin{split}
\EE[Y(t)] & = i^{-1} \frac{\partial \psi_{\mathbf{C}(t)}(\mathbf{u})}{\partial u_2} \Big \vert_{\mathbf{u} = \mathbf{0}} = \int_0^t [p_2(v)+2p_3(v)] r S(v)dv \\
\EE[Y(t)^2] & = i^{-2} \frac{\partial^2 \psi_{\mathbf{C}(t)}(\mathbf{u})}{\partial^2 u_2} \Big \vert_{\mathbf{u}= \mathbf{0}} \\
& = \int_0^t [p_2(v)+4p_3(v)] r S(v)dv + \Big[\int_0^t [p_2(v)+2p_3(v)] r S(v)dv\Big]^2 \\
\Var[Y(t)] & = \EE[Y(t)^2] - (\EE[Y(t)])^2 = \int_0^t [p_2(v)+4p_3(v)] r S(v)dv
\end{split}
\end{equation}

\begin{equation}
\begin{split}
\frac{\partial \psi_{\mathbf{C}(t)}(\mathbf{u})}{\partial u_3} &= \exp \Big\{ \int_0^t [(e^{iu_1}-1)p_1(v) + (e^{iu_2}-1)p_2(v) + (e^{-iu_1+2iu_2}-1)p_3(v) + (e^{iu_3}-1)p_4(v)]  rS(v)dv \Big\} \\
&\int_0^t ie^{iu_3}p_4(v)  rS(v)dv \\
\frac{\partial^2 \psi_{\mathbf{C}(t)}(\mathbf{u})}{\partial^2 u_3} &= \exp \Big\{ \int_0^t [(e^{iu_1}-1)p_1(v) + (e^{iu_2}-1)p_2(v) + (e^{-iu_1+2iu_2}-1)p_3(v) + (e^{iu_3}-1)p_4(v)]  rS(v)dv \Big\} \\
&\int_0^t i^2e^{iu_3}p_4(v)  rS(v)dv +\\
&\exp \Big\{ \int_0^t [(e^{iu_1}-1)p_1(v) + (e^{iu_2}-1)p_2(v) + (e^{-iu_1+2iu_2}-1)p_3(v) + (e^{iu_3}-1)p_4(v)]  rS(v)dv \Big\} \\
& \Big[\int_0^t i e^{iu_3}p_4(v) rS(v)dv \Big]^2
\end{split}
\end{equation}

\begin{equation}
\begin{split}
\EE[Z(t)] & = i^{-1} \frac{\partial \psi_{\mathbf{C}(t)}(\mathbf{u})}{\partial u_3} \Big \vert_{\mathbf{u} = \mathbf{0}} = \int_0^t p_4(v) r S(v)dv \\
\EE[Z(t)^2] & = i^{-2} \frac{\partial^2 \psi_{\mathbf{C}(t)}(\mathbf{u})}{\partial^2 u_3} \Big \vert_{\mathbf{u}= \mathbf{0}} \\
& = \int_0^t p_4(v) r S(v)dv + \Big[\int_0^t p_4(v) r S(v)dv\Big]^2 \\
\Var[Z(t)] & = \EE[Z(t)^2] - (\EE[Z(t)])^2 = \int_0^t p_4(v) r S(v)dv
\end{split}
\end{equation}

Using the second derivatives, we also obtain the covariances
\begin{equation}
\begin{split}
\frac{\partial^2 \psi_{\mathbf{C}(t)}(\mathbf{u})}{\partial u_1 \partial u_2} &= \exp \Big\{ \int_0^t [(e^{iu_1}-1)p_1(v) + (e^{iu_2}-1)p_2(v) + (e^{-iu_1+2iu_2}-1)p_3(v) + (e^{iu_3}-1)p_4(v)]  rS(v)dv \Big\} \\
& \int_0^t [-2i^2 e^{-iu_1 + 2iu_2} p_3(v)]  rS(v)dv +\\
& \exp \Big\{ \int_0^t [(e^{iu_1}-1)p_1(v) + (e^{iu_2}-1)p_2(v) + (e^{-iu_1+2iu_2}-1)p_3(v) + (e^{iu_3}-1)p_4(v)]  rS(v)dv \Big\} \\
& \int_0^t [ie^{iu_2}p_2(v) +2 ie^{-iu_1 + 2iu_2}p_3(v)]  rS(v)dv \int_0^t [ie^{iu_1}p_1(v) - ie^{-iu_1 + 2iu_2}p_3(v)] rS(v)dv \\
\EE [X(t)Y(t)] & = i^{-2} \frac{\partial^2 \psi_{\mathbf{C}(t)}(\mathbf{u})}{\partial u_1 \partial u_2}\Big \vert_{\mathbf{u} = \mathbf{0}} \\
&=
-\int_0^t 2 p_3(v)]  rS(v)dv + \int_0^t [p_1(v)-p_3(v)]rS(v) dv \int_0^t[p_2(v)+2p_3(v)]rS(v)dv \\
Cov [X(t)Y(t)] &=\EE [X(t)Y(t)] - E[X(t)]E[Y(t)] = -\int_0^t 2 p_3(v)]  rS(v)dv
\end{split}
\end{equation}

\begin{equation}
\begin{split}
\frac{\partial^2 \psi_{\mathbf{C}(t)}(\mathbf{u})}{\partial u_1 \partial u_3} & = \exp \Big\{ \int_0^t [(e^{iu_1}-1)p_1(v) + (e^{iu_2}-1)p_2(v) + (e^{-iu_1+2iu_2}-1)p_3(v) + (e^{iu_3}-1)p_4(v)]  rS(v)dv \Big\} \\
& \int_0^t ie^{iu_3}p_4(v)  rS(v)dv \int_0^t [ie^{iu_1}p_1(v) - ie^{-iu_1 + 2iu_2}p_3(v)] rS(v)dv \\
\EE [X(t)Z(t)] & = i^{-2} \frac{\partial^2 \psi_{\mathbf{C}(t)}(\mathbf{u})}{\partial u_1 \partial u_3}\Big \vert_{\mathbf{u} = \mathbf{0}} \\
&=
 \int_0^t [p_1(v)-p_3(v)]rS(v) dv \int_0^t p_4(v) rS(v)dv \\
Cov [X(t)Z(t)] &=\EE [X(t)Z(t)] - E[X(t)]E[Z(t)] = 0
\end{split}
\end{equation}

\begin{equation}
\begin{split}
\frac{\partial^2 \psi_{\mathbf{C}(t)}(\mathbf{u})}{\partial u_2 \partial u_3} & = \exp \Big\{ \int_0^t [(e^{iu_1}-1)p_1(v) + (e^{iu_2}-1)p_2(v) + (e^{-iu_1+2iu_2}-1)p_3(v) + (e^{iu_3}-1)p_4(v)]  rS(v)dv \Big\} \\
& \int_0^t ie^{iu_3}p_4(v)  rS(v)dv \int_0^t [ie^{iu_2}p_2(v) + 2ie^{-iu_1 + 2iu_2}p_3(v)]  rS(v)dv \\
\EE [Y(t)Z(t)] & = i^{-2} \frac{\partial^2 \psi_{\mathbf{C}(t)}(\mathbf{u})}{\partial u_2 \partial u_3}\Big \vert_{\mathbf{u} = \mathbf{0}} \\
&=
 \int_0^t [p_2(v)+2p_3(v)]rS(v) dv \int_0^t p_4(v) rS(v)dv \\
Cov [Y(t)Z(t)] &=\EE [Y(t)Z(t)] - E[Y(t)]E[Z(t)] = 0
\end{split}
\end{equation}
\end{document}



\begin{equation}
\begin{split}
E\Big[e^{iu \frac{Y(t)}{F(t)}}\Big] & = \exp \left\{  \int_0^t \Big[ e^{i \frac{u}{\mu_{y_t}}}p_2(v) - p_2(v) + e^{2i \frac{u}{\mu_{y_t}}}p_3(v) - p_3(v)\Big] r S(v) dv\right\} \\
& = \exp \left\{ \int_0^t \Big[ \Big( 1 + \frac{iu}{\mu_{y_t}}  + o(\mu^{-1}_{y_t}) \Big) p_2(v) - p_2(v)+ \Big( 1 + \frac{2iu}{\mu_{y_t}}  + o(\mu^{-1}_{y_t}) \Big) p_3(v) - p_3(v)  \Big] r S(v) dv\right\} \\
& = \exp \left\{ \frac{iu}{\mu_{y_t}} \int_0^t  [p_2(v) +2 p_3(v)] r S(v) dv  +  o (\mu^{-1}_{y_t})\int_0^t [p_2(v) + 2p_3(v)] r S(v) dv \right\} \\
& \rightarrow \exp \left\{iu\right\}.
\end{split}
\end{equation}

\begin{equation}
\begin{split}
E \Big[e^{iu \frac{Y(t)}{\sigma_{y_t}}} \Big] &= \exp \left\{  \int_0^t \Big[ e^{i \frac{u}{\sigma_{y_t}}}p_2(v) - p_2(v) + e^{i \frac{2u}{\sigma_{y_t}}}p_3(v) - p_3(v)\Big] r S(v) dv\right\} \\
& = \exp \left\{ \int_0^t \Big[ \Big( 1 + \frac{iu}{\sigma_{y_t}} + \frac{i^2 u^2}{2 \sigma^2_{y_t}} + o(\sigma^{-2}_{y_t}) \Big) p_2(v) - p_2(v)+ \\
& \Big( 1 + \frac{2iu}{\sigma_{y_t}} + \frac{4 i^2 u^2}{2 \sigma^2_{y_t}} + o(\sigma^{-2}_{y_t}) \Big) p_3(v) - p_3(v)  \Big] r S(v) dv\right\} \\
& = \exp \left\{ \frac{iu}{\sigma_{y_t}}\int_0^t  [p_2(v) + 2p_3(v)] r S(v) dv +\frac{i^2u^2}{2\sigma^2_{x_t}} \int_0^t  [p_1(v) + 4p_3(v)] r S(v) dv \\
& +  o (\sigma^{-2}_{x_t})\int_0^t [p_1(v) + p_3(v)] r S(v) dv \right\} \\
& \rightarrow \exp \left\{\frac{iu}{\sigma_{y_t}} F(t)  - \frac{u^2}{2} \right\} \\
E \Big[ e^{iu \frac{Y(t)-F(t)} {\sigma_{y_t}}} \Big] & = E \Big[e^{iu \frac{Y(t)}{\sigma_{y_t}}} \Big] \cdot e^{-iu \frac{F(t)}{\sigma_{y_t}}} \rightarrow e^{- \frac{u^2}{2 }}.
\end{split}
\end{equation} 

Therefore, using substitution with $q = P(v) = \int_0^v [p_1(u) - p_3(u)] du$ and $dq/dv = [p_1(v) - p_3(v)]$, we get
\begin{equation*}
\EE[X(t)] = S_0 r \int_0^{P(t)} \exp\{rq\}dq = S_0 \exp \{rq\} \Big \vert_0^{P(t)} = S_0 \exp\{r P(t)\} - S_0.
\end{equation*}
Thus, $\EE[S_0+X(t)] = S(t)$. In similar fashion, one can show that $\EE[Y(t)]=F(t)$.