\documentclass[10pt]{article}
\usepackage[hmargin={0.6in, 0.6in}, vmargin={0.9in, 0.9in}]{geometry}
\usepackage{amsmath, amsthm, booktabs}
\usepackage{amssymb}
\usepackage{pdfpages}
\usepackage{setspace}
\usepackage{booktabs}
\usepackage{graphicx}

%\usepackage{fancyheadings}
%\usepackage{hyperref}

\renewcommand{\baselinestretch}{1}

\newtheorem{proposition}{Proposition}

\newcommand{\dif}{\mathrm{d}}
\newcommand{\Var}{\mathbb{V}}
\newcommand{\PP}{\mathbb{P}}
\newcommand{\EE}{\mathbb{E}}

\title{Likelihood for the Branching Process Model for Stem Cell Proliferation}
\author{The Author}
\date{} % Activate to display a given date or no date (if empty),
         % otherwise the current date is printed

\begin{document}
\maketitle
Suppose the process starts at time $T_0 = 0$ and the number of starting stem cell $S_0$. The interarrival time of the next event is exponentially distributed
$$\Delta_{T_i} = T_i - T_{i-1} \sim \text{Exp}(r\cdot S_{i-1}), \quad i = 1, \cdots, n,$$
where $r$ is the division rate. At the event time $T_i$, the pair of random variable $(X_i, Y_i)$ has the following distribution 
\begin{equation}
(X_i, Y_i, Z_i) = \begin{cases}
(+1, 0, 0), & p_1(T_i) \\
(0, +1, 0), & p_2(T_i) \\
(-1,+2, 0), & p_3(T_i) \\
(0, 0, +1), & p_4(T_i)
\end{cases} 
\end{equation}

Assume that there is no dud stem cells (i.e $p_4(t) = 0 \forall t$, and we observe all the events (both the time of the events $T_0, T_1, \cdots, T_n$ and the number of stem cells $S_0, S_1, \cdots, S_n$. Since we observe every events, we can know how the cell changes $(X_i,Y_i)$ at each event time.  We define
$$\alpha_{T_i}(S_i) = P(T_1, T_2, \cdots, T_i, S_1, \cdots, S_i).$$
We show that $\alpha_{T_i} (S_i)$ could be computed with dynamic programming. 
\begin{equation}
\begin{split}
\alpha_{T_1}(S_1) & = P(T_1, S_1) \\
& = P(S_1 \vert T_1) \cdot P(T_1) \\
& = P(X_1 = S_1 -S_0 \vert T_1) \cdot P(\Delta_{T_1} = T_1 - T_0) \\
& = [p_1(T_1)\cdot I_{(X_1 = 1)} + p_2(T_1)\cdot I_{(X_1 = 0)} + p_3(T_1) \cdot I_{(X_1=-1)}] \cdot [rS_0 e^{-rS_0 \Delta_{T_1}}].
\end{split} 
\end{equation}

\begin{equation}
\begin{split}
\alpha_{T_2} (S_2) & = P(T_1, T_2, S_1, S_2) \\
& = P(T_2, S_2 \vert T_1, S_1) \cdot P(T_1, S_1) \\
& = P(S_2 \vert T_2, T_1, S_1) \cdot P(T_2 \vert T_1, S_1) \cdot P(T_1, S_1) \\
& = P(X_2 = S_2 -S_1 \vert T_2, T_1, S_1) \cdot P(\Delta_{T_2} = T_2 - T_1 \vert T_1, S_1) \cdot P(T_1, S_1) \\
& =  [p_1(T_2)\cdot I_{(X_2 = 1)} + p_2(T_2)\cdot I_{(X_2 = 0)} + p_3(T_2) \cdot I_{(X_2=-1)}] \cdot  [rS_1 e^{-rS_1 \Delta_{T_2}}] \cdot \alpha_{T_1}(S_1).
\end{split}
\end{equation}

\begin{equation}
\begin{split}
\alpha_{T_i} & = P(T_1, \cdots, T_i, S_1, \cdots S_i) \\
& = P(T_i, S_i \vert T_1, \cdots T_{i-1}, S_1 \cdots S_{i-1}) \cdot P(T_1, \cdots T_{i-1}, S_1 \cdots S_{i-1}) \\
& = P(S_i \vert T_1, \cdots T_{i-1}, T_i, S_1 \cdots S_{i-1}) \cdot P(T_i \vert T_1, \cdots T_{i-1}, S_1 \cdots S_{i-1}) \cdot P(T_1, \cdots T_{i-1}, S_1 \cdots S_{i-1}) \\
& = P(X_i = S_i-S_{i-1} \vert T_i, S_{i-1}) \cdot P(\Delta_{T_i} = T_i - T_{i-1} \vert T_{i-1}, S_{i-1}) \cdot P(T_1, \cdots T_{i-1}, S_1 \cdots S_{i-1}) \\
& =  [p_1(T_i)\cdot I_{(X_i = 1)} + p_2(T_i)\cdot I_{(X_i = 0)} + p_3(T_i) \cdot I_{(X_i=-1)}] \cdot  [rS_{i-1} e^{-rS_{i-1} \Delta_{T_i}}] \cdot \alpha_{T_{i-1}}(S_{i-1}).
\end{split}
\end{equation}

\end{document}

